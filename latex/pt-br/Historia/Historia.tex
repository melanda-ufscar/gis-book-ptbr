\chapter{História dos SIG}

\pagestyle{fancy}

O desenvolvimento dos SIG desde sua origem até os dias atuais é notável. A popularização das tecnologias e os esforços de desenvolvimento realizados por um amplo leque de ciências beneficiárias dos SIG contribuíram para redefinir a disciplina e incorporar elementos que, na época, eram impensáveis.

Podemos situar a origem dos SIG no início da década de \textbf{1960}, como resultado da convergência de dois fatores principais: a \textbf{necessidade crescente de informação geográfica} e sua gestão eficiente, e o \textbf{surgimento dos primeiros computadores}.

As bases para o futuro surgimento dos SIG encontram-se alguns anos antes dessa década, com o desenvolvimento de novos enfoques em cartografia, como a \textbf{geografia quantitativa}, que antecipavam necessidades futuras que a informatização traria.

A primeira experiência relevante que combina geografia e informática ocorre em 1959, quando Waldo Tobler define os princípios de um sistema chamado MIMO (map in--map out), com o objetivo de aplicar os computadores ao campo da cartografia. Nesse sistema, ele estabelece os princípios básicos para a criação, codificação, análise e representação de dados geográficos em um ambiente computacional.

O primeiro Sistema de Informação Geográfica formalmente desenvolvido surge no Canadá. Esse sistema, chamado CGIS (Canadian Geographical Information Systems), foi desenvolvido no início da década de 1960 por Roger Tomlinson, amplamente conhecido como o <<pai do SIG>>.

Em meados dos anos 60, os aplicativos SYMAP e GRID estabelecem, respectivamente, as bases dos dois principais enfoques para manipulação de informação geográfica: o enfoque \textbf{raster} e o enfoque \textbf{vetorial}. Ambos serão explicados em detalhes mais adiante neste livro. Os conceitos fundamentais para análise raster foram estabelecidos pouco depois por Dana Tomlin, ao desenvolver a chamada \textbf{álgebra de mapas}.

Durante os anos 1960, os SIG se desenvolvem a partir desses elementos iniciais e passam a ser incorporados à comunidade cartográfica, deixando de ser apenas uma ferramenta experimental.

A evolução dos SIG percorre, desde então, várias etapas, avançando rapidamente sob influência de diversos fatores externos. Essa evolução ocorre na própria disciplina dos SIG, nas tecnologias de apoio, nos dados e nas técnicas e formulações utilizadas.

\section{A evolução dos SIG como disciplina}

Inicialmente, os SIG eram apenas uma combinação de elementos da cartografia quantitativa aliados aos sistemas computacionais da época. Eram um campo dominado por cartógrafos e geógrafos que buscavam adaptar seus conhecimentos às tecnologias emergentes. Com o tempo, no entanto, os SIG passaram a incorporar um \textbf{grande número de outras disciplinas}, cujas contribuições e influências se tornaram tão relevantes quanto — ou até mais — que as da cartografia ou geografia.

Coincidindo com os primeiros estágios de desenvolvimento dos SIG, surge uma preocupação crescente com o meio ambiente, o que favorece o desenvolvimento das ciências relacionadas — a maioria das quais se tornariam usuárias diretas dos SIG. O SIG passa a ser gradualmente integrado às atividades de \textbf{gestão ambiental}, como apoio essencial à sua análise.

No início da década de 1970, já é evidente que os SIG são ferramentas promissoras, e surgem não apenas esforços para seu desenvolvimento e consolidação, mas também conferências, simpósios e a inclusão da temática em \emph{currículos} universitários. Nos anos 1980, consolidam-se revistas e fóruns especializados que levarão a disciplina a um público mais amplo.

No setor comercial, a indústria dos SIG também se consolida nos anos 70. A \textbf{ESRI} (Environmental Systems Research Institute), empresa pioneira e líder do setor até hoje, é fundada em 1969, e seus produtos são fundamentais na transformação dos SIG em ferramentas amplamente adotadas. O primeiro SIG de código aberto, \textbf{GRASS} (Geographic Resources Analysis Support System), surge em 1985.

O maior avanço na incorporação dos SIG a contextos não profissionais ocorre na primeira década do século XXI, com o surgimento de serviços de mapeamento como o \textbf{Google Maps}. A popularização dos \textbf{navegadores GPS}, que incorporam elementos de representação e análise típicos dos SIG, é outro exemplo da sua disseminação junto ao público geral.

\section{A evolução da tecnologia}

Três são os principais blocos de desenvolvimento tecnológico que mais influenciaram os Sistemas de Informação Geográfica:

\begin{itemize}
 \item \textbf{Saídas gráficas.} A evolução das capacidades gráficas — intensa desde o início e ainda em curso — foi acompanhada de perto pelos SIG, que passaram a incorporar progressivamente melhorias tanto na visualização em tela quanto na geração de mapas impressos.

 \item \textbf{Armazenamento e acesso aos dados.} O crescimento do volume de dados geográficos exigiu avanços em capacidade de armazenamento e velocidade de leitura, para garantir uma operação fluida dos sistemas.

 \item \textbf{Entrada de dados.} Nos primeiros anos dos SIG, os dados geográficos eram obtidos de documentos em papel, digitalizados manualmente e armazenados em cartões perfurados. Desde esses sistemas mecânicos até os equipamentos atuais — com \emph{scanners} de alta precisão e técnicas de digitalização automatizada — o processo de entrada de dados evoluiu radicalmente.
\end{itemize}

Além desses fatores, a evolução dos computadores impactou todos os elementos de \emph{software}. Passou-se dos grandes computadores aos computadores pessoais, e os programas — como os próprios SIG — também fizeram essa transição.

A partir do final dos anos 1980, a elaboração e análise de cartografia se torna possível em PCs de baixo custo, afastando-se das grandes máquinas e sistemas dedicados de alto custo.

A evolução das plataformas segue adiante. Atualmente, há uma tendência de levar os SIG para \textbf{plataformas móveis}, como celulares e tablets, especialmente úteis para coleta de dados em campo. A integração com tecnologias de posicionamento global, como o GPS, torna essa prática ainda mais eficaz.

O surgimento da Internet transformou toda a sociedade, incluindo o campo dos SIG. O primeiro uso relevante é registrado em 1993 com o \emph{Xerox PARC}, o primeiro \textbf{servidor de mapas}. Em 1994, é lançado o primeiro atlas digital online — o Atlas Nacional do Canadá. Mais recentemente, os conceitos da Web 2.0 são aplicados aos SIG, possibilitando o surgimento do \textbf{Web Mapping}.

\section{A evolução dos dados}

As primeiras bases de dados geográficas continham \textbf{mapas digitalizados} ou elementos vetoriais derivados desses. A partir daí, novas fontes de dados com estruturas mais adequadas ao processamento digital começaram a surgir, criando uma relação bidirecional entre SIG e dados.

Um marco importante é o lançamento dos primeiros \textbf{satélites de observação da Terra}. As técnicas de fotografia aérea, originalmente militares e iniciadas no século XIX com balões, passam a ser utilizadas em escala global com os satélites. Em 1980, é criada a SPOT, primeira empresa comercial a fornecer imagens de satélite para toda a superfície terrestre.

As \textbf{tecnologias de posicionamento e localização} também representam fontes primárias de dados. O sistema GPS torna-se totalmente operacional em 1981, e em 2000 sua precisão é ampliada para uso civil.

Como as aplicações, os tipos de dados geográficos digitais tornam-se mais populares e recebem maior atenção. Em 1976, o USGS (Serviço Geológico dos EUA) publica os primeiros \textbf{Modelos Digitais de Elevação} (MDE). Em 2000, são divulgados os dados da \emph{Shuttle Radar Topographic Mission} (SRTM), cobrindo 80\% da superfície terrestre com resolução de aproximadamente 30 metros.

O surgimento de novas tecnologias, como o \textbf{LiDAR}, possibilita níveis de precisão antes inatingíveis e abre novas possibilidades de análise.

A evolução dos dados não é apenas técnica, mas também \textbf{social e organizacional}. Passa-se a entender a importância de estratégias adequadas para a gestão de dados espaciais, surgindo as chamadas \textbf{Infraestruturas de Dados Espaciais} (IDE). O maior exemplo é a NSDI, dos EUA (1994). Na Europa, destaca-se a diretiva INSPIRE, de 2007.

Muitos desses avanços seguem as diretrizes do \emph{Open GIS Consortium} (OGC), fundado em 1994 para \textbf{padronizar} o uso e disseminação de dados geográficos.

\section{A evolução das técnicas e formulações}

Após os primeiros SIG serem implementados, surgem novas técnicas e abordagens que ampliam as possibilidades de análise.

Um dos primeiros antecedentes da análise espacial ocorre em 1854, quando John Snow utiliza mapas de pontos para localizar a origem de um surto de cólera na Inglaterra — um exemplo clássico de cartografia analítica.

Em seu livro \emph{Design with Nature} (1969), Ian McHarg define os elementos fundamentais da \textbf{sobreposição e combinação de mapas} — base do funcionamento das \emph{camadas} em um SIG.

Destaca-se também o desenvolvimento do \textbf{análise do relevo}, que se transforma com o SIG. A antiga orografia dá lugar a uma ciência quantitativa centrada na análise morfométrica.

Outros elementos da cartografia evoluem da mesma forma. Em 1819, Pierre Charles Dupin cria o primeiro \textbf{mapa de coropletas}, que se tornaria uma representação comum com os SIG.

O avanço em sistemas de desenho assistido por computador (CAD) e gráficos digitais impulsiona a \textbf{geometria computacional}, base do modelo vetorial nos SIG e também das representações gráficas.

\pagestyle{empty}
