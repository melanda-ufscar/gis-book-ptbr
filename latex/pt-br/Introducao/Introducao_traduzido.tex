
\chapter{¿Qué es un SIG?}

\pagestyle{fancy}

La mayor parte de la información que manejamos en cualquier tipo de disciplina está georreferenciada. Es decir, se trata de información a la cual puede asignarse una posición geográfica, y es por tanto información que viene acompañada de otra información adicional relativa a su localización. 

Un \textbf{Sistema de Información Geográfica} (SIG) es una herramienta para trabajar con información georreferenciada. En particular, un SIG es un sistema que permite la realización de las siguientes operaciones:

\begin{itemize}
	\item \textbf{Lectura, edición, almacenamiento} y, en términos generales, \textbf{gestión} de datos espaciales.
	\item \textbf{Análisis} de dichos datos. Esto puede incluir desde consultas sencillas a la elaboración de complejos modelos, y puede llevarse a cabo tanto sobre la \textbf{componente espacial} de los datos (la localización de cada valor o elemento) como sobre \textbf{la componente temática} (el valor o el elemento en sí).
	\item Generación de \textbf{documentos} tales como mapas, informes, gráficos, etc.
\end{itemize}


Un SIG representa un paso más allá de los mapas clásicos. Mientras que un mapa es una representación de un conjunto de datos espaciales, y aunque esta representación resulta de enorme importancia, en el entorno de un SIG no es sino un elemento más de un conjunto de ellos. El SIG no incluye solo los datos y la representación, sino también las operaciones que pueden hacerse sobre el mapa, que no son ajenas a este sino partes igualmente de todo ese sistema.

Un SIG es una herramienta versátil y de amplio alcance, y hoy día la gran mayoría de disciplinas se benefician del uso de SIG de uno u otro modo. Una de las principales razones por las que esto sucede es el \textbf{carácter integrador} de los SIG. Los siguientes son algunos de los contextos principales en los que un SIG ejerce tal función integradora.

\begin{itemize}
\item \textbf{SIG como integrador de información}. Un nexo común entre la mayoría de disciplinas es el hecho de que sus objetos de estudio están asociados a una localización en el espacio. Esto va a permitir combinarlas y obtener resultados a partir de un análisis común. El SIG es, en este contexto, el marco necesario en el que incorporar esa información georreferenciada y trabajar con ella.

\item \textbf{SIG como integrador de tecnologías}. Una gran parte de las tecnologías que han surgido en los últimos años (y seguramente de las que surjan en los próximos) se centran en el aprovechamiento de la información espacial, y están conectadas en mayor o menor medida a un SIG para ampliar su alcance y sus capacidades. Por su posición central en el conjunto de todas las tecnologías, los SIG cumplen además un papel de unión entre ellas, conectándolas y permitiendo una relación fluida alrededor de las funcionalidades del propio SIG.

\item \textbf{SIG como integrador de personas}. Las funciones básicas que un SIG ha de cumplir cubren en realidad un rango amplio de trabajo, y engloban las necesidades de usuarios que con anterioridad no tenían entre sí un marco de trabajo común tan definido. Esto tiene como consecuencia que existe una mejor coordinación entre ellos, pues es la propia herramienta quien establece las características de la relaciones existentes, y estas no dependen ya únicamente del propio ámbito de aplicación.

\item \textbf{SIG como integrador de teorías y fundamentos}. En un principio, podemos entender un SIG como la unión de dos ciencias: la geografía y la informática. Sin embargo, un análisis más detallado nos revela que un SIG incorpora elementos de muchas ciencias distintas, como pueden ser las disciplinas relacionadas con la tecnología y el manejo de información (informática, diseño de bases de datos, tratamiento digital de imágenes), las dedicadas al estudio de la Tierra desde un punto de vista físico (geología,  oceanografía, ecología), las dedicadas al estudio de la Tierra desde un punto de vista social y humano (antropología, geografía, sociología), las dedicadas al estudio del entendimiento humano (ciencias del conocimiento, psicología) o las disciplinas que tradicionalmente han realizado una integración de conocimientos de otros ámbitos distintos, entre las que cabe destacar la geografía.

El término \textbf{geomática}, formado a partir de los vocablos \emph{geografía} e \emph{informática}, se emplea con frecuencia para hacer mención a todo ese grupo de ciencias relacionadas con los SIG. 
\end{itemize}


Con todo lo anterior, se tiene que SIG es un sistema que integra tecnología informática, personas e información geográfica, y cuya principal función es capturar, analizar, almacenar, editar y representar datos georreferenciados. 

Desde otro punto de vista, un SIG puede considerarse compuesto de cinco bloques fundamentales.

\begin{itemize}
 \item \textbf{Datos.} Los datos son necesarios para hacer que el resto de componentes de un SIG cobre sentido y puedan ejercer su papel en el sistema. La información geográfica, la verdadera razón de ser los SIG, reside en los datos, y es por ello que el conocimiento exhaustivo de los datos, su naturaleza, su procedencia, su calidad, así como su gestión y almacenamiento, resulta obligado para una buena comprensión los propios SIG.

\item \textbf{Análisis.} El análisis es una las funcionalidades básicas de los SIG, y una de las razones fundamentales que llevaron al desarrollo de estos. Un ordenador es una herramienta con enorme capacidad de cálculo, y esta puede aplicarse a los datos espaciales para obtener resultados de muy diversa índole.

En mayor o menor medida, un SIG siempre incorpora una serie de formulaciones que permiten la obtención de resultados y el análisis de los datos espaciales. Las ventajas de la incorporación de todos estos procesos en una única herramienta, el SIG, van desde la \textbf{automatización de tareas} a la aparición de nuevos procesos que producen resultados que no podrían ser obtenidos de otro modo. 

\item \textbf{Visualización}. Cualquier tipo de información puede ser representada de forma gráfica, lo cual facilita la interpretación de dicha información o parte de esta. En el caso particular de la información geográfica, la visualización no solo es una forma más de trabajar con esa información, sino que resulta la forma principal, por ser aquella a la que estamos más acostumbrados gracias al uso de mapas.

Al contrario que un mapa, que de por sí es de naturaleza gráfica, en un SIG trabajamos con datos de tipo puramente numérico. Para poder presentar una utilidad similar a la de un mapa, un SIG debe incluir capacidades que generen representaciones visuales a partir de esos datos numéricos.

La visualización de la información geográfica se rige por los mismos conceptos y principios que se emplean para la confección de cartografía impresa, y estos deben ser conocidos por el usuario de SIG, ya que una de las tareas de este es el diseño cartográfico y las preparación de los elementos de visualización para poder realizar su trabajo sobre las representaciones creadas. 

\item \textbf{Tecnología.} Se incluyen en este elemento tanto el \emph{hardware} sobre el que se ejecutan las aplicaciones SIG, como dichas aplicaciones, es decir el \emph{software} SIG. Además de la propia plataforma, el \emph{hardware} incluye una serie de periféricos habituales en el trabajo con SIG, como son los periféricos para entrada de datos geográficos y creación de cartografía. 

\item \textbf{Factor organizativo.} Engloba los elementos relativos a la coordinación entre personas, datos y tecnología, o la comunicación entre ellos, entre otros aspectos. Con la creciente complejidad de los SIG, la gestión de las relaciones entre sus elementos es cada vez más importante.
\end{itemize}

A lo largo de este libro, detallaremos cada uno de estos bloques en los capítulos correspondientes.




\pagestyle{empty}