
\chapter{O que é um SIG?}

\pagestyle{fancy}

A maior parte das informações que manipulamos em qualquer tipo de disciplina está georreferenciada. Ou seja, trata-se de informação à qual pode ser atribuída uma posição geográfica, e que, portanto, vem acompanhada de dados adicionais relativos à sua localização.

Um \textbf{Sistema de Informação Geográfica} (SIG) é uma ferramenta para trabalhar com informações georreferenciadas. Em particular, um SIG é um sistema que permite a realização das seguintes operações:

\begin{itemize}
	\item \textbf{Leitura, edição, armazenamento} e, em termos gerais, \textbf{gestão} de dados espaciais.
	\item \textbf{Análise} desses dados. Isso pode incluir desde consultas simples até a elaboração de modelos complexos, e pode ser realizado tanto sobre a \textbf{componente espacial} dos dados (a localização de cada valor ou elemento), quanto sobre a \textbf{componente temática} (o valor ou o elemento em si).
	\item Geração de \textbf{documentos} como mapas, relatórios, gráficos etc.
\end{itemize}

Um SIG representa um avanço em relação aos mapas clássicos. Enquanto um mapa é uma representação de um conjunto de dados espaciais — representação essa de enorme importância —, no ambiente de um SIG ele é apenas um dos elementos do sistema. O SIG inclui não só os dados e sua representação, mas também as operações que podem ser realizadas sobre eles, que fazem parte integrante desse sistema.

O SIG é uma ferramenta versátil e de amplo alcance, e atualmente a grande maioria das disciplinas se beneficia do seu uso de alguma forma. Uma das principais razões para isso é o \textbf{caráter integrador} dos SIG. Abaixo estão alguns dos contextos principais em que o SIG exerce tal função integradora:

\begin{itemize}
\item \textbf{SIG como integrador de informações}. Um ponto comum entre muitas disciplinas é o fato de que seus objetos de estudo estão associados a uma localização no espaço. Isso permite combiná-los e obter resultados por meio de uma análise conjunta. Nesse contexto, o SIG é o ambiente necessário para incorporar essa informação georreferenciada e trabalhar com ela.

\item \textbf{SIG como integrador de tecnologias}. Muitas das tecnologias que surgiram nos últimos anos (e certamente muitas das que ainda surgirão) estão focadas no aproveitamento da informação espacial, e estão conectadas, em maior ou menor grau, a um SIG para ampliar seu alcance e capacidades. Por sua posição central entre essas tecnologias, os SIG também atuam como elo entre elas, conectando-as e permitindo uma interação fluida por meio de suas funcionalidades.

\item \textbf{SIG como integrador de pessoas}. As funções básicas que um SIG deve cumprir abrangem uma ampla gama de atividades e atendem às necessidades de usuários que anteriormente não possuíam um ambiente de trabalho comum tão bem definido. Isso resulta em melhor coordenação entre esses usuários, pois é a própria ferramenta que define as características das relações estabelecidas, deixando de depender exclusivamente do contexto de aplicação.

\item \textbf{SIG como integrador de teorias e fundamentos}. Inicialmente, podemos entender um SIG como a união entre duas ciências: a geografia e a informática. Contudo, uma análise mais aprofundada revela que o SIG incorpora elementos de diversas outras áreas, como as relacionadas à tecnologia e ao tratamento da informação (informática, banco de dados, processamento digital de imagens), ciências que estudam a Terra sob uma perspectiva física (geologia, oceanografia, ecologia), ciências humanas e sociais (antropologia, geografia, sociologia), ciências do conhecimento e cognição (psicologia, epistemologia), além das disciplinas que tradicionalmente integram saberes de diferentes domínios — com destaque para a própria geografia.

O termo \textbf{geomática}, formado a partir das palavras \emph{geografia} e \emph{informática}, é frequentemente utilizado para se referir a esse conjunto de ciências relacionadas aos SIG.
\end{itemize}

Com base em tudo isso, entende-se que um SIG é um sistema que integra tecnologia da informação, pessoas e dados geográficos, cuja principal função é capturar, analisar, armazenar, editar e representar dados georreferenciados.

Sob outra perspectiva, um SIG pode ser considerado composto por cinco blocos fundamentais:

\begin{itemize}
 \item \textbf{Dados.} Os dados são essenciais para que os demais componentes do SIG façam sentido e possam cumprir seu papel no sistema. A informação geográfica — razão de ser dos SIG — está contida nos dados. Por isso, conhecer detalhadamente sua natureza, origem, qualidade, bem como sua gestão e armazenamento, é indispensável para compreender adequadamente o funcionamento dos SIG.

 \item \textbf{Análise.} A análise é uma das funcionalidades básicas dos SIG, e uma das principais razões que motivaram seu desenvolvimento. O computador é uma ferramenta com grande capacidade de processamento, e isso pode ser aplicado aos dados espaciais para gerar resultados dos mais diversos tipos.

 Em algum grau, todo SIG incorpora procedimentos que permitem obter resultados a partir da análise dos dados espaciais. As vantagens de incluir esses processos em uma única ferramenta — o SIG — vão desde a \textbf{automação de tarefas} até a criação de novos processos, cujos resultados não poderiam ser obtidos de outra forma.

 \item \textbf{Visualização.} Qualquer tipo de informação pode ser representada graficamente, o que facilita sua interpretação. No caso específico da informação geográfica, a visualização não é apenas uma forma adicional de trabalhar com os dados — é a principal, pois estamos habituados a isso por meio dos mapas.

 Diferentemente de um mapa, que é por natureza gráfico, um SIG lida com dados puramente numéricos. Para apresentar uma utilidade semelhante à de um mapa, o SIG deve incluir recursos para gerar representações visuais a partir desses dados.

 A visualização de dados geográficos segue os mesmos princípios utilizados na cartografia impressa, e esses princípios devem ser dominados pelo usuário do SIG, uma vez que ele será responsável pelo design cartográfico e pela preparação dos elementos visuais para trabalhar com as representações criadas.

 \item \textbf{Tecnologia.} Este componente inclui tanto o \emph{hardware} onde as aplicações SIG são executadas, quanto o \emph{software} SIG em si. Além da plataforma, o \emph{hardware} também abrange periféricos comuns no trabalho com SIG, como os utilizados para entrada de dados geográficos e produção cartográfica.

 \item \textbf{Fator organizacional.} Refere-se aos aspectos ligados à coordenação entre pessoas, dados e tecnologia, bem como à comunicação entre esses elementos. Com a crescente complexidade dos SIG, a gestão das interações entre seus componentes torna-se cada vez mais importante.
\end{itemize}

Ao longo deste livro, detalharemos cada um desses blocos nos capítulos correspondentes.

\pagestyle{empty}

