\chapter*{Prólogo}

Hace ahora más de cinco anos que se publicó la primera versión de \textit{Sistemas de Información Geográfica}, un libro libre sobre fundamentos de SIG en español, y apenas unos meses desde que apareció la segunda. El libro ha tenido una acogida excelente, y mi intención es seguir manteniéndolo actualizado en la medida que sea posible, reflejando los avances que, a buen seguro, van a producirse en el campo de los SIG.

Existe, no obstante, un obstáculo importante para que el libro alcance a todos los públicos: su tamaño. Por su completitud, y por la complejidad propia de la disciplina, el libro es un volumen de más de 800 páginas cargadas de detalle. La segunda versión se presenta en un único tomo, frente a los dos en que consistía la primera, pero aún así sigue quedando como una obra de consulta demasiado extensa para leerse de principio a fin. Para el lector que comienza a introducirse en el ambito de los SIG y no busca especializarse, resulta un volumen intimidante y es, no hay duda, difícil de abordar.

Este libro intenta ser una alternativa a la obra completa, de tal forma que resulte más accesible para quienes desean tener una perspectiva global de la disciplina de los SIG, sin entrar en detalles demasiado específicos. Es, basicamente, una versión resumida de aquel, pensada con la idea de usarse no como libro de consulta, sino como libro de lectura. Además de ser más breve, se presenta en un formato más adecuado para esta clase de propósito, con algunas modificaciones en su enfoque y con menos contenido gráfico.

He respetado en líneas generales la estructura de los capítulos, de modo que es fácil para el lector que desee profundizar en uno de ellos encontrar este en el libro completo. Desde ese punto de vista, puede entenderse este libro como una especie de <<índice>> de su hermano mayor, un índice, no obstante, prolijo y con suficiente información como ofrecer al lector una visión detallada del mundo de los SIG.

Este es también, por supuesto, un libro libre, que espero que progrese de una forma dinámica gracias a la contribución de sus lectores. Si encuentras cualquier error o quieres colaborar en mejorar estas páginas, no dudes en escribirme a \\ \texttt{volayaf@gmail.com}.